%!TEX root = main.tex

\section{Conclusions}
\label{sec:conclusions}

In this paper, we studied the problem of maximizing earnings of 
    drivers employed by ride-sharing platforms like Uber, Lyft, etc. 
Our work confirms the idea that even in a high-dimensional and big data 
    domain such as ride-sharing, the inherent structure of the data can
    be leveraged to develop a simple, interpretable, fair and highly efficient
    framework that aims to achieve this goal.
Extensive simulations based on New York City taxi 
    datasets showed that our framework is easy to calibrate due to its
    robustness to imperfections in hyperparameter tuning.
Our experiments provided evidence for the differential impact of the platform's
    objectives on driver earnings.
Finally, we demonstrated that our model generalizes well to fluctuations 
    in supply and demand.
We make available an OpenAI gym environment for comparative studies.

% In the future, we would like to study problems related to the impact of autonomous electric vehicles 
%     on ride-sharing fleets. 
% In that setting, the optimal strategies to match such vehicles to ride requests 
%     would have to take into account the inherent slow battery charging processes. 
% This will bring in a new \textit{charge-scheduling} aspect to an already 
%     multi-faceted problem. 
% Moreover, it is unlikely that the entire human workforce would be replaced by 
%     autonomous vehicles instantly. 
% It will be interesting  to study fair compensation strategies for human
%     drivers in light of the gradual integration of autonomous and conventional vehicles.
