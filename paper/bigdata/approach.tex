%!TEX root = main.tex

\section{Learning Framework}
\label{sec:approach}
In this section, we describe our approach for solving the \textsc{MaxEarnings} 
    problem.
Our method is a \emph{model-based reinforcement learning} approach, and its 
    description is provided in Algorithm~\ref{alg:general_learning}. 
 
As with any reinforcement learning approach, we train our model by
    allowing the drivers to repeatedly interact with an environment 
    in form of the city's ride demand data from a representative day.
Each interaction, which is $T$ timesteps long, constitutes an \emph{episode}
    of the training process.
Each episode constitutes of 3 phases described below.

%!TEX root = main.tex

\begin{algorithm}
Initialization $Q_I(t, h, a) \leftarrow 0, Q_C(t, h, a) \leftarrow 0, \coordination(t, h) \leftarrow 0$\;
\For{each episode $e = 1, \cdots, E$}{
    \For{each time step $t = 1, \cdots, T$}{
        \For{each driver $i = 1, \cdots, n$}{
            Generate two random numbers $\eta_0, \eta_1 \in [0,1]$\;
            \uIf{$\eta_0 \leq \epsilon$}{
                Choose exploratory action\;
            }
            \Else{
                \uIf{$\eta_1 \leq \coordination(t, h_i)$}{
                $a$ = Independent action $a^*$ from $Q_I$\;}
                \Else{
                $a$ = Coordinated action $a^c$ from $Q_C$\;
            }
            }
            Receive reward $E(t, h_i, a)$\;
            Compute rebalance matrix $\bf Z$\;
        }
    }
    \For{each zone $h \in \hexbins$}{
        $\forall t, a$ update $Q_I(t, h, a)$ \;
        $\forall t$ update degree of coordination $\coordination(t, h)$ \;
        $\forall t, a$ update $Q_C(t, h, a)$ \;
    }
}
\caption{General learning approach}
\label{alg:general_learning}
\end{algorithm}


\spara{Exploratory phase (lines 5-7): }
During this phase of the algorithm, drivers exhibit an
    \emph{exploratory behavior} by choosing a pseudo-random action 
    with a probability $\epsilon$.
These randomly chosen actions allow the model to explore a larger portion of
    the policy space, preventing its policy from converging to a local
    minimum.
This is similar to the $\epsilon$-greedy behavior of
    Q-learning~\cite{Sutton1998-wd}.

\spara{Exploitative phase (lines 9-12): }
During this phase of the algorithm, the rest of the drivers
    exhibit an \emph{exploitative behavior} using the policy learned up until 
    the previous episode of training.
The policy recommends exploitative actions to individual drivers based upon 
    the time of the day and their locations, independently of each other,
    henceforth referred to as \emph{independent actions}.
However, certain recommended actions may result in supply-demand imbalances
    when a large number of drivers relocate to the same city zone with an
    insufficient demand, or too few of them relocate to a zone with excess 
    demand.
We postulate that explicit coordination is essential to prevent such
    supply-demand imbalances from occurring.
Hence, we introduce the \emph{degree of coordination ({\coordination})} - a 
    probabilistic value that signifies the extent to which drivers located in 
    the same city zone need to coordinate their actions.
Whenever a zone has a positive degree of coordination, the exploitative actions
    recommended to a {\coordination} fraction of drivers in the zone are derived
    from solving a reward-maximizing linear program, henceforth referred to as
    \emph{coordinated actions}.

It should be noted that it is the explicit criterion for recommending a 
    coordinated action that sets our
    approach apart from recent works in the field of deep reinforcement
    learning across different applications and domains.

\spara{Learning phase (lines 15-18): }
Actions recommended in the exploratory and exploitative phases of
    the episode result in drivers picking up passengers or relocating themselves
    to different city zones, thereby observing rewards of their actions (line 13).
The learning phase of the algorithm computes a rebalancing matrix (line 14) to use
    in conjunction with the observed rewards to further improve upon the
    policy.

Having developed an intuition for the major building blocks of
    Algorithm~\ref{alg:general_learning}, we now explain these phases in 
    greater detail. 

\subsection{Exploratory phase}
Over the course of training, when a driver located in 
    zone $h$ chooses to explore, we model the probability of driver's 
    exploratory ride distance using a Gaussian function with a random variable 
    $K_{\geq 0}$. 
Specifically, the probability that a driver relocates to a zone at distance 
    $k \geq 0$ is given by:
    $Pr[K = k] = be^{-\frac{k^2}{2c^2}}$.

After sampling an exploration distance $k$, the driver chooses the actual 
    destination by sampling uniformly at random from all zones at 
    distance $k$. 
When $k=0$, the driver chooses to wait in the current zone, while for $k >0$, 
    the driver chooses a relocate action. 
The experiments in this paper were all conducted using $b=0.7$ and
    $c=1$ (chosen via grid-search), allowing explorations up to 3 hexagonal zones away.
In contrast,~\cite{Lin2018-vs} restricts drivers to single zone distance
    relocations, reducing their ability to learn policies that mitigate 
    supply-demand imbalances by relocating supply from zones further away in a
    single timestep.
% Given a sufficient number of training episodes, the framework is not
%     sensitive to the value of these parameters as they primarily affect the speed of
%     learning by controlling exploration distance. 
Over the course of training, $\epsilon$ is annealed exponentially from
    1 to 0, thereby outputting an entirely exploitative model at the end of
    the training.

\subsection{Exploitative phase}
Exploitative behavior is manifested in the form of independent
    actions (line 10) and coordinated actions (line 12) when the degree of
    coordination is positive.
We detail these next.

\subsubsection{Choice of independent action}
For each independent action chosen by a driver, we record the  
    reward earned. 
This reward is then used to update the \emph{value} of the action for the 
    next episode, based on the learning rate ($\alpha$) and the
    discount factor ($\gamma$). 
These values are stored in a Q-table denoted by 
    $Q_I \in \mathbb{R}^{T \times m \times |\textbf{A}|}$.
For each zone $h$, at time $t$, the best independent action ($a^*$) 
 is chosen by (line 10, Algorithm~\ref{alg:general_learning})
    \begin{equation*}
        a^*(t, h) = \argmax_{a \in \actionmatrix_h}Q_I(t, h, a), 
    \end{equation*}
    where $\actionmatrix_h$ refers to the $h$-th row of {\actionmatrix}.

\subsubsection{Independent learning}
Based upon the observations of drivers undertaking independent actions (both
    exploratory and exploitative), we
    update the independent learning matrix ($Q_I$) as
    described below.

% \spara{Updating $Q_I$ for wait actions:}
% Let $\waitingdrivers$ denote the set of drivers choosing wait action at time $t$
%     while located in zone $h$. 
% A successful wait may lead a driver $i \in \waitingdrivers$ to destination $h'$
%     with net earnings $E\big(t, h, a(h, h)\big) = r(h, h')$ and a travel time 
%     of $\tau(h, h')$. 
% An unsuccessful wait will force the driver to stay in zone $h$ with net earnings
%     $E\big(t, h, a(h, h)\big) = 0$ and travel time $\tau(h,h) = 1$.
% The utility of the wait action is therefore
%     \begin{equation*}
%         \mathcal{U}_{(t, h,h)} = \sum_{i \in \waitingdrivers} \bigg[E\big(t, h_i,
%             a(h_i,h_i)\big) + \gamma \times
%         Q_I(t'_i, h'_i, a^*(t'_i, h'_i))\bigg],
%     \end{equation*}
%     where $t'_i = t_i + \tau(h_i, h'_i)$ and we discount the future rewards with 
%     a factor $\gamma$.
% We use the utility of the wait action to update the entry $Q_I(t, h, h)$ as follows: 
%     \begin{equation}
%     \label{eq:Q_I_update_wait}
%         Q_I(t, h, h) \leftarrow (1 - \alpha)Q_I(t, h, h) +
%         \frac{\alpha}{|\waitingdrivers|}\mathcal{U}_{(t, h,h)}.
%     \end{equation}
% Normalizing the update term by the number of drivers choosing the wait action 
%     captures the average utility of the wait action.
% The term $Q_I(t, h, h)$ on the right hand side of the equation denotes the 
%     values learned upto the previous episode of the training.

\spara{Updating $Q_I$ for wait actions:}
Let $\waitingdrivers_{(h, h')}$ denote the number of drivers choosing to wait
    in zone $h$ at time $t$, and ending up in zone $h'$. 
A successful wait generates net earnings $E\big(t, h, a(h, h)\big) = r(h, h')$ 
    and consumes a travel time $\tau(h, h')$, while 
    an unsuccessful wait i.e., $h' = h$, generates zero net earnings
    and consumes one timestep.
The utility of the wait action is therefore
    \begin{equation*}
        \mathcal{U}_{(t, h,h)} = \sum_{h'} 
        \waitingdrivers_{(h, h')}
        \bigg[E\big(t, h,
            a(h,h)\big) + \gamma 
    Q_I(t', h', a^*(t', h'))\bigg]
    \end{equation*}
%%  JB -- removed trailing comma -- in other column
    where $t' = t + \tau(h, h')$ and we discount the future rewards with 
    a factor $\gamma$.
We use the utility of the wait action to update the entry $Q_I(t, h, h)$ as follows: 
    \begin{equation}
    \label{eq:Q_I_update_wait}
        Q_I(t, h, h) \leftarrow (1 - \alpha)Q_I(t, h, h) +
        \frac{\alpha}{\sum_{h'}\waitingdrivers_{(h, h')}}\mathcal{U}_{(t, h,h)}.
    \end{equation}
Normalizing the update term by the number of drivers choosing the wait action 
    captures the average utility of the wait action.
The term $Q_I(t, h, h)$ on the right hand side of the equation denotes the 
    values learned upto the previous episode of the training, and $\alpha$ is 
    the learning rate.

\spara{Updating $Q_I$ for relocate actions:}
Let $\relocatingdrivers_{(h, h')}$ denote the number of drivers relocating from zone $h$
    to zone $h'$. The utility of such relocation is given by
    \begin{equation*}
        \mathcal{U}_{(t,h,h')} = \relocatingdrivers_{(h, h')} \bigg[E(t, h,
            a(h, h')) + \gamma
        Q_I(t', h', a^*(t', h'))\bigg],
    \end{equation*}
    where $t' = t + \tau(h, h')$. 
We use the utility of the relocate actions to update the entry $Q_I(t, h, h')$
    of the independent table as follows: 
    \begin{equation}
    \label{eq:Q_I_update_relocate}
        Q_I(t, h, h') \leftarrow (1 - \alpha)Q_I(t, h, h') 
        + \frac{\alpha}{\relocatingdrivers_{(h, h')}}\mathcal{U}_{(t, h,h')}.
    \end{equation}
Using Equations~\eqref{eq:Q_I_update_wait} and ~\eqref{eq:Q_I_update_relocate}, the
    $Q_I$ matrix is updated in line 16 of Algorithm~\ref{alg:general_learning} 
    using the evidence obtained via simulations in form of 
    utilities $\mathcal{U}_{(t, h, h')}$ of both the wait and relocate actions.

\subsubsection{Choice of coordinated action}
The choice of coordinated action is more intricate and non-standard, and we 
    next explain it in detail.
To guide the coordinated behavior of drivers in line 12 of
    Algorithm~\ref{alg:general_learning}, we solve a reward-maximizing
    rebalancing operation between city zones experiencing supply-demand
    imbalances. 
There are two principal components driving the coordinated behavior: 
\emph{degree of coordination} ({\coordination}) which controls the need of 
    coordination in a particular zone at a time, and \emph{coordinated learning
    matrix} ($Q_C$) which determines the choice of action as a response to the 
    need of coordination.
Thus, each coordinated action is associated with a probability for it to participate in
    the rebalancing operation that is stored in the matrix $Q_C$. 
Note that $Q_C$ contains learned probabilities, as against the usual 
    action-value nature of $Q_I$. 

Let the policy learned at the end of $k$-th episode during training be denoted
    by ${\policy}_k$. 
Following this policy induces a driver supply $S^{\policy_k}$ during 
    the $(k+1)$-th episode of training. 
For each zone $h$, at time $t$, the coordinated action ($a^c$) in line 12 of 
    Algorithm~\ref{alg:general_learning} is obtained by uniformly sampling 
    from the probability vector $Q_C(t, h)$.

\spara{Imbalance matrix (\imbalancematrix):}
A matrix ${\imbalancematrix} \in \relocatingdrivers^{|\timesteps| \times
    m}$ such that each entry $\delta(t,h)$ denotes the supply-demand imbalance
    experienced at zone $h$ at time $t$ during the $(k+1)$-th episode. 
Specifically, each entry of the imbalance matrix can be given by, $\delta(t, h) =
    S^{\policy_k}(t, h) - \sum_{h'} d_t(h, h')$.
We mask the imbalance matrix using an imbalance threshold parameter 
    {\imbalancethreshold} such that,
    \begin{equation*}
        \delta(t,h) = 
            \begin{cases}
            \delta(t, h) &\text{ if } \big|\delta(t, h)\big|
            \geq \imbalancethreshold \\
            0 &\text{ otherwise. }
        \end{cases}
    \end{equation*}
Using this parameter allows us to control the level of imbalances that the
    framework should attempt to mitigate. 

\spara{Rebalancing graph (\rebalancinggraph):}
Based upon the supply-demand imbalance matrix induced at the end of an
    episode, we create the \emph{rebalancing graph}
    $\rebalancinggraph = (V,E)$ consisting of
    imbalanced zones as nodes and edges as corresponding relocation actions between
    them. %in order to mitigate those imbalances.
This is a bipartitle graph with nodes 
%% JB: set -- redundant
$V = \big\{V_+ \cup V_-\big\}$ where $V_+$ is the set
    of nodes with excess supply, i.e., $\delta(t,h) > 0$ and $V_-$ is the
    set of nodes with supply deficit, i.e., $\delta(t, h) < 0$.
Thus each node $v_i \in V$ in the rebalancing graph is associated with 
    three attribues: \textit{imbalanced zone} $(v_i^h)$, \textit{time of
    imbalance} $(v_i^t)$ and \textit{magnitude of imbalance} $\big(\delta(v_i^t, v_i^h)\big)$.
The edge set $E$ consists of directed edges from the nodes in $V_+$ 
    to nodes in $V_-$ and they model feasible relocations. 
Thus:
    $E = \big\{ e_{ij}: v_i \in V_+, v_j \in V_-, v_i^t + \tau(v_i^h,v_j^h) \leq v_j^t\big\}$.
% The travel-time constraint limits the edges to ones from
%     supply excess nodes to supply deficit nodes such that a relocating driver
%     from the excess node will be able to reach the deficit node in time to mitigate
%     the deficit. 
The travel time constraint filters out edges where a relocating driver from
    an excess supply node cannot reach the deficit node in time.
Each edge $e_{ij}$ is associated with utility:
    \begin{equation*}
        \mathcal{U}_{(i,j)} = \underbrace{Q_I(v_j^t, v_j^h ,v_j^h )}_\text{wait
        action at $v_j^h$} - \underbrace{\textrm{cost}(v_i^h,v_j^h)}_\text{relocation cost} -
    \underbrace{Q_I(v_i^t, v_i^h, v_i^h).}_\text{wait action at $v_i^h$}
    \end{equation*}
Thus, the utility of an edge measures the net value for a driver relocating along it
    during coordinated behavior.

\spara{Rebalancing operation:}
Given a rebalancing graph {\rebalancinggraph}, we wish to relocate drivers from
    supply excess zones to supply deficit zones.  
We aim to find a matching that maximizes the net reward of all relocations, 
    in order to maximize the driver earnings.  
Such a rebalancing operation can be achieved by solving a
    \emph{Minimum Cost Flow} problem expressed in the form of the linear program below.
    \begin{eqnarray*}
        \text{maximize } &\sum_{e_{ij} \in E} f_{ij} \times \mathcal{U}_{(i,j)} \nonumber \\
        \text{s.t., }& \nonumber\\
        \forall e_{ij} \in E, &f_{ij} \geq 0 \nonumber\\
        \forall v_i \in V_+, &\sum_{v_j \in V_-} f_{ij} \leq \delta(v_i^t, v_i^h)
        \nonumber\\
        \forall v_j \in V_-, &\sum_{v_i \in V_+} f_{ij} \leq |\delta(v_j^t, v_j^h)|
    \end{eqnarray*}
\noindent Here, we calculate the number of
    excess drivers who should relocate from an excess node to a deficit node and store
    it in the form of a flow vector $f \in \relocatingdrivers^{|E|}$ indexed along the
    edges set such that $f_{ij}$ denote the flow from $v_i$ to $v_j$. 

If the platform aims to maximize demand fulfillment, we can formulate it
    as a \emph{Maximum Flow} problem
by setting the utility associated with each edge 
    $\mathcal{U}_{(i,j)}=1$.

As the constraint matrices -- in both problems -- are unimodal, the solutions 
    of the linear programs are integral flow vectors and are thus optimal. 
Note that the size of the constraint matrix increases with a decrease in 
    the {\imbalancethreshold} parameter. 
However, we can greatly reduce the sizes of corresponding linear programs and 
    hence the computation time by solving a set of smaller linear programs; 
    one for each connected component of the rebalancing graph.

\subsubsection{Coordinated learning}
Based upon the computed imbalance matrix ({\imbalancematrix}) and the solution
    to the rebalancing operation above, we are now in a position to
    update the coordinated learning matrix ($Q_C$) and the degrees of coordination
    ({\coordination}) as described below.
It should be noted that while the choice of coordinated action from the matrix 
    $Q_C$ is
    influenced by the reward-maximizing rebalancing described above, the degree
    of coordination {\coordination} is merely influenced by the supply-demand
    imbalances induced as a result of the policy learnt so far.

\spara{Updating $Q_C$ for rebalancing operation:}
We capture the rebalancing operation in form of a rebalance matrix 
    $\rebalancematrix \in \mathbb{R}^{|\timesteps|
    \times m \times m}$ where each entry $\zeta(t, h, h')$ denotes a probability of
    a rebalancing relocation from zone $h$ to zone $h'$ being required at time $t$.
For every edge $e_{ij} \in E$, we update {\rebalancematrix} as follows,
    \begin{eqnarray*}
        \zeta(v_i^t, v_i^h, v_i^h) &=& 
            \frac{\delta(v_i^t, v_i^h) -\sum_{v_j \in V_-} f_{ij}}{\delta{v_i^t,
            v_i^h}} \nonumber\\
        \zeta(v_i^t, v_i^h, v_j^h) &=& \frac{f_{ij}}{\delta(v_i^t, v_i^h)}.
    \end{eqnarray*}
Using the rebalance matrix, we update $Q_C$ in line 18 of
    Algorithm~\ref{alg:general_learning} as follows,
    \begin{equation}
    \label{eq:Q_C_update}
        Q_C(t, h, h') \leftarrow (1 - \alpha)Q_C(t, h, h') + \alpha \zeta(t,
        h, h').
    \end{equation}

\spara{Updating degree of coordination ({\coordination}):}
At the end of each training episode $(k+1)$, we use the realized imbalance
    matrix ({\imbalancematrix}) to determine the degree of coordination required 
    within each zone at
    every time step.  
We update the degree of coordination as follows:
    \begin{equation}
    \label{eq:doc_update}
        \coordination_{k+1}(t, h) = (1 - \alpha)\coordination_{k}(t, h) + \alpha\mu(t, h),
    \end{equation}
    where the rebalancing raio $\mu$ is computed as:
    \begin{equation*}
        \mu(t, h) = 
        \begin{cases}
        \frac{\delta(t,h)}{S^{\policy_k}(t, h)} &\text{if }
            \delta(t,h) > 0. \\ 
            \frac{\big|\delta(t, h)\big|}{\sum_{h'}d_t(h,h')} &\text{if }
            \delta(t,h) < 0\text{ and } \coordination_{k}(t, h) > 0. 
        \end{cases}
    \end{equation*}
While the former condition encourages driver relocations in zones with supply excess, 
    the latter condition discourages it in zones with supply deficit.
Thus, we use Equation~\eqref{eq:doc_update} to update the degree of coordination
    for each zone in line 17 of Algorithm~\ref{alg:general_learning}.
\iffalse
\subsection{Discussion}

We conclude this section by highlighting some of the features of our approach 
    that make it appealing to use in practice. 

First of all, over the course of training, our approach learns by trialing driver
    actions over historically observed demand data and recommends 
    strategic relocations to drivers when there is enough evidence to do so.
This is done \emph{proactively}, i.e., with the goal of preventing such an
    imbalance from actually occurring. 
One should contradict this with other reinforcement
    learning based approaches~\cite{Lin2018-vs, Li2019-bp} that try to resolve the
    imbalance issues \emph{ex-post}, 
    or dynamic pricing based approaches which assume full knowledge of future
    demand~\cite{Ma2018-hb}.

On deployment, our model relies only on trends learned from the 
    historical data.
This design decision is motivated by the empirically observed strong periodicity in demand.
It makes our model relatively parameter-free, thereby providing robustness to
    demand perturbations.
This decision is validated by the model generalizability experiment in
    Section~\ref{sec:model_generalizability}, where we show that the recommendations
    made by our algorithm are robust to the presence of perturbations in the demand.
In contrast, deep reinforcement learning based approaches~\cite{Mnih2013-sj,Tang2019-xu,Lin2018-vs, Wen2017-vp, 
    Wang2018-bv} require as inputs full knowledge of supply and demand 
    distribution during deployment. 
The stochastic
    gradient-descent algorithm used during training of networks uniformly
    samples experiences observed under previous training policies, making it
    impossible to reliably trace back and explain the actions recommended
    to the drivers.
Furthermore, this increases the sensitivity of model 
    performance to the tuning of numerous hyperparameters, making them 
    difficult to deploy in the real world.

A key characteristic of our approach is the explicit coordination achieved
    by solving a \emph{minimum cost flow} problem, allowing all our
    recommendations to be easily traced back and explained.
\fi
