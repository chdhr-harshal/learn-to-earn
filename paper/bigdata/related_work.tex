%!TEX root = main.tex
\section{Related Work}
\label{sec:related_work}

In this section, we discuss existing work in comparison to ours.

\spara{Driver recommender systems:}
The problem of spatio-temporal demand prediction 
    to inform an idle taxi driver of favorable locations for passenger
    pickups had been studied extensively even before the advent of ride-hailing
    services. 
These works focus on the case of a self-interested individual acting in isolation.
For instance, Li et al.~\cite{Li2011-qm} use a
    large-scale taxi GPS trace dataset to identify salient features associated
    with successful passenger pickup locations, while in two separate studies, 
    Yuan et al.~\cite{Yuan2011-pd, Yuan2013-gj} develop a recommender
    system to guide both idle taxi drivers and waiting passengers to convenient
    locations in order to optimize social welfare.
More recently, Chaudhari et al.~\cite{Chaudhari2018-cv} devise driver-oriented strategies to
    recommend favorable driving schedules and pickup locations
    to optimize the earnings of an individual on-demand ride-sharing driver.
In contrast to our approach, the recommender systems in these works 
    are agnostic to driver interactions
    and may result in unfavorable supply excesses in certain locations
    when adopted by many drivers simultaneously.

\spara{Capacity repositioning systems:}
Traditional works in driver dispatch systems~\cite{Lee2004-hh, Zhang2016-vz,
    Seow2010-qg} typically rely on queuing-theoretical models to asymptotically
    optimize supply-demand matching while reducing congestion-related issues.
Aided by the precise estimates of supply and
    demand in real-time, recent approaches~\cite{Xu2018-xb, Zhang2017-id} 
    leverage
    demand volume and ride destination forecasting models to use in
    combinatorial optimization techniques.
These approaches do not scale well to use cases on contemporary platforms,
    where fleets of as many as 10,000 drivers serve a single city. 

    
More recent work~\cite{Mnih2013-sj,Tang2019-xu,
    Lin2018-vs, Wen2017-vp, Wang2018-bv} 
    addresses the scalability issue by using deep reinforcement learning to learn 
    control policies in high-dimensional input space. 
While effective in this high-volume data domain, these methods rely on
    external proprietary models to generate inputs for the driver dispatch
    systems.
For example, the approach of ~\cite{Lin2018-vs} heavily relies on 
    a proprietary simulator built by Didi Chuxing in order to generate inputs 
    required during model training, making it impossible 
    to reproduce their results for comparison purposes.
To the best of our knowledge, building such a simulator is itself an 
   active research problem.
Moreover, deep-learning based techniques suffer from a lack of
    explainability.
Cognizant of these issues, our approach does not rely on any proprietary models 
    but rather learns high-quality solutions \emph{from scratch} based solely 
    upon historically observed data.
Moreover, we achieve that without sacrificing the explainability of the
    model.
In the absence of the need for coordination, our model assumes homogeneity of 
    drivers in the same location and provides
    \emph{envy-free} recommendations while also making it scalable. 
Furthermore, we have publicly made available our entire codebase and
    the reinforcement learning environment required to reproduce every result 
    presented in this work, thus 
    enabling future comparative studies~\cite{github-page}.

From the learning point of view, our approach not only learns \emph{how} to coordinate 
    but also \emph{when} it is required to do so. 
This is achieved by augmenting  vanilla reinforcement 
    learning (in the form of tabular Q-learning) with
    combinatorial techniques to aid the rebalancing of driver
    distribution.

\spara{Platform economics:}
Studies of ride-hailing services as multi-sided economic marketplaces have 
    investigated the impacts of the platform's pricing policies on the platform profits, 
    the consumer surplus, and the driver
    wages~\cite{Castillo2018-he, Bimpikis2016-yf, Besbes2019-ds, Ma2018-hb}.
S{\"u}hr et al.~\cite{Tom_Suhr-ps} investigate fairness in driver earnings 
    distribution using driver-passenger matchings optimized to attain 
    income equality goals.
Recently, Chen et al.~\cite{Chen2019-li} combines platform economics with the capacity
    repositioning problem using a contextual bandit framework. 
There is a growing body of literature studying the interplay between platform
    pricing and strategic driver behaviors, for which we refer the readers to
    ~\cite{Yan2018-wq}.
Our work contributes to this domain by 
    developing a scalable framework that can be used to verify the results of
    asymptotic dynamic pricing models via realistic simulations.

