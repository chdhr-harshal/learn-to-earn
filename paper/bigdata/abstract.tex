%!TEX root = main.tex

\begin{abstract}
The problem of optimizing social welfare objectives on multi-sided ride-hailing 
    platforms such as Uber, Lyft, etc., is challenging, due to misalignment
    of objectives between drivers, passengers, and the platform itself.
An ideal solution aims to minimize the response time for each hyperlocal 
    passenger ride request, while simultaneously maintaining high demand
    satisfaction and supply utilization across the entire city. 
Economists tend to rely on dynamic pricing mechanisms that stifle
    price-sensitive excess demand and resolve supply-demand imbalances 
    that emerge in specific neighborhoods.
In contrast, computer scientists primarily view it as a demand prediction 
    problem with the goal of preemptively repositioning supply to such
    neighborhoods using \emph{black-box} coordinated multi-agent 
    deep reinforcement learning-based approaches.
Here, we introduce explainability in the existing supply-repositioning 
    approaches by establishing the need for coordination between the drivers 
    at specific locations and times.
Explicit need-based coordination allows our framework to use a
    simpler \emph{non-deep} reinforcement learning-based approach,
    thereby enabling it to explain its recommendations \emph{ex-post}.
Moreover, it provides \emph{envy-free} recommendations i.e., drivers
    at the same location and time do not envy one another's expected 
    future earnings.
Our experimental evaluation demonstrates the effectiveness, robustness, and 
    generalizability of our framework. 
Finally, in contrast to previous works, we make available a reinforcement 
    learning environment for \emph{end-to-end} reproducibility of our work and 
    to encourage future comparative studies.
\end{abstract}
